\documentclass[a4paper,14pt]{article}
\usepackage{graphicx}
\usepackage{hyperref}
\usepackage[margin=1in]{geometry}
\usepackage{multicol}
\usepackage{fancyhdr}
\usepackage{setspace}
\usepackage{tcolorbox}
\usepackage{array}

% Document starts here
\begin{document}

% University Logo and Header
\begin{center}
    \includegraphics[width=0.3\textwidth]{xyz.png} % Placeholder for university logo
    \vspace{0.3cm}
\end{center}

\vspace{0.8cm}

% Lab Notebook Details
\begin{center}
    {\LARGE \textbf{Lab Notebook}} \\
      \vspace{0.2cm}
    {\LARGE \textbf{Team - 33}} \\
    \vspace{0.50cm}
    \textbf{UNIVERSITY NAME} \\
    \vspace{0.20cm}
    \begin{tabular}{|p{0.50\linewidth}|}
    \hline
    Maulana Abul Kalam Azad University of Technology \\
    \hline
    \end{tabular}
    \vspace{0.50cm}
    
    \textbf{ASSIGNMENT DETAILS} \\
    \vspace{0.5cm}
    \begin{tabular}{|p{0.95\linewidth}|}
    \hline
     Topic: \textbf{Create a Git Repository Containing Lab Notebook in a LaTeX File} \\
     Subject: \textbf{Software Tools and Technologies} \\
     Team no: \textbf{33} \\
    \textbf{GitHub Repo Link:} \href{https://github.com/nasim888/LabNotebook_Group33}{https://github.com/nasim888/LabNotebook_Group33} \\
    \hline
    \end{tabular}
\end{center}

\vspace{1cm}

% Team Members Section
\begin{center}
    \textbf{TEAM MEMBERS}
\end{center}

\vspace{0.5cm}

\begin{multicols}{2}

    \begin{tcolorbox}[colframe=black, colback=white, sharp corners=southwest, boxrule=0.5mm, arc=5mm]
    \textbf{Member 1 (Lead):} \newline
    Name: Nasim Ahemed \newline
    Reg No: 233002410597 \newline
    Course: BSc in IT (Cyber Security) \newline
    GitHub: \href{https://github.com/nasim888}{github.com/nasim888}
    \end{tcolorbox}

    \begin{tcolorbox}[colframe=black, colback=white, sharp corners=southwest, boxrule=0.5mm, arc=5mm]
    \textbf{Member 2:} \newline
    Name: Sk Wasim Afrose \newline
    Reg No: 233001010501 \newline
    Course: BCA \newline
    GitHub: \href{https://github.com/WasimAfrose}{github.com/WasimAfrose}
    \end{tcolorbox}

    \begin{tcolorbox}[colframe=black, colback=white, sharp corners=southwest, boxrule=0.5mm, arc=5mm]
    \textbf{Member 3:} \newline
    Name: Pronab Sarkar \newline
    Reg No: 233002410615 \newline
    Course: BSc in IT (Data Science) \newline
    GitHub: \href{https://github.com/pronab12}{github.com/pronab12}
    \end{tcolorbox}

    \columnbreak

    \begin{tcolorbox}[colframe=black, colback=white, sharp corners=southwest, boxrule=0.5mm, arc=5mm]
    \textbf{Member 4:} \newline
    Name: Sneha Ghosh \newline
    Reg No: 233002410557 \newline
    Course: BSc in IT (Artificial Intelligence) \newline
    GitHub: \href{https://github.com/snehaghosh0}{github.com/snehaghosh0}
    \end{tcolorbox}

    \begin{tcolorbox}[colframe=black, colback=white, sharp corners=southwest, boxrule=0.5mm, arc=5mm]
    \textbf{Member 5:} \newline
    Name: Arnab Bhowmick \newline
    Reg No: 233002410577 \newline
    Course: BSc in IT (Artificial Intelligence) \newline
    GitHub: \href{https://github.com/arnab438}{github.com/arnab438}
    \end{tcolorbox}

\end{multicols}
\newpage

% Acknowledgment Section
\begin{center}
    \hrule height 0.5pt
    \vspace{0.2cm}
    {\LARGE \textbf{ACKNOWLEDGMENT}} \\
    \vspace{0.2cm}
    \hrule height 0.5pt
\end{center}

\vspace{1cm}

We, the members of this group, would like to extend our heartfelt gratitude to everyone who contributed to the successful completion of this project.

\vspace{0.5cm}

First and foremost, we would like to thank our instructors, \textbf{Ayan Ghosh, Pabitra Pal}, whose guidance and support were invaluable throughout the course of this assignment. Their insights and feedback helped us improve the quality of our work.

\vspace{0.5cm}

We also wish to acknowledge our teammates, \textbf{Nasim Ahemed, Sk Wasim Afrose, Pronab Sarkar, Sneha Ghosh, Arnab Bhowmick} for their dedication, hard work, and collaboration, which made this project possible. The diverse skills and perspectives each of you brought to the table enriched our discussions and results.

\vspace{0.5cm}

Additionally, we extend our appreciation to \textbf{Chatgpt, Youtube} for providing resources and assistance that greatly enhanced our project.

\vspace{0.5cm}

Lastly, we would like to thank our friends and team members for their continuous encouragement and understanding during the entire process of completing this project.

\vspace{0.5cm}

This project would not have been possible without the contributions of all these individuals.

\vspace{1cm}

\begin{flushleft}
\textbf{Thank you!} \\
\textbf{Team - 33 Members} \\
\textbf{Sep 22, 2024}
\end{flushleft}

\vspace{2cm}

\begin{flushright}
\rule{5cm}{0.5pt} \\
\textbf{Signatures}
\end{flushright}

\newpage

% Index Section
\begin{center}
    \hrule height 0.5pt
    \vspace{0.3cm}
    {\LARGE \textbf{INDEX}} \\ 
    \vspace{0.3cm}
    \hrule height 0.5pt
\end{center}

\vspace{1cm}

\begin{center}
    \begin{tabular}{|>{\centering\arraybackslash}p{1.5cm}|p{10.5cm}|>{\centering\arraybackslash}p{3cm}|}
        \hline
        \vspace{0.1cm}
        \textbf{SL. NO.} & \vspace{0.1cm} \textbf{PROGRAM NAME} & \vspace{0.1cm} \textbf{TEACHERS SIGNATURE} \vspace{0.1cm} \\ 
        \hline
        \vspace{0.5cm}
        \textbf{1.} & \vspace{0.5cm} Introduction to Github and Github desktop version installation \vspace{0.5cm} &  \\ 
        \hline
        \vspace{0.5cm}
        \textbf{2.} & \vspace{0.5cm} Create a local repository. Build a C programme of calculator in the local repository, commit and publish it as a public repository. \vspace{0.5cm} &  \\ 
        \hline
        \textbf{3.} & \vspace{0.5cm} Cloning Already Made Repo \& in That Repo converting Submit button to Chin Tapak Dum Dum \vspace{0.5cm} &  \\ 
        \hline
        \vspace{0.5cm}
        \textbf{4.} & \vspace{0.5cm} Creating LaTeX Repository on GitHub \vspace{0.5cm} &  \\ 
        \hline
        \vspace{0.5cm}-
        \textbf{5.} & \vspace{0.5cm} Create a CV using LaTeX document \vspace{0.5cm} &  \\ 
        \hline
    \end{tabular}
\end{center}

\newpage

\pagestyle{fancy}
\fancyhf{}
\fancyfoot[C]{\thepage}

% Title formatting
  {\normalfont\fontsize{24}{30}\bfseries}{}

\begin{center}
    \Huge \textbf{LAB NOTEBOOK ENTRIES} \\
    \vspace{10pt}
    \normalsize Each member create one lab notebook entry and commit their Work in Repo \\
    \vspace{30pt}
    
    \Large \textbf{Entry by: Nasim Ahamed} \\
    \normalsize Date: [September 22, 2024] \\
    \end{center}
    \vspace{30pt}
    
    \Large 
    \textbf{Lab Assignment 1:} \\
    Introduction to Github and Github desktop version installation \\
    \vspace{40pt}
    
    \begin{center}
    % Placeholder for the image
    \includegraphics[width=0.3\textwidth]{github.png} \\ % Add your image here
    \vspace{20pt}
    
    \begin{tabular}{|p{0.40\linewidth}|}
    \hline
    \huge \textbf{What Is GitHub} \\
    \hline
    \end{tabular}
    \end{center}
    \vspace{20pt}
    
    \normalsize 
    GitHub is a web-based platform used for version control and collaborative software development. 
    It allows multiple developers to work on a project simultaneously, track changes in the code, and collaborate effectively. 
    GitHub uses Git, a distributed version control system, to manage and store code in repositories. 
    It offers additional features like pull requests, issue tracking, and team collaboration tools.
    


\newpage

\section*{Key Features of GitHub:}
\begin{itemize}
    \item \textbf{Repositories (Repos):} A GitHub repository is where the code and project files are stored. Each repo contains the project history and can be cloned, forked, or contributed to by others.
    \item \textbf{Branches:} A branch in GitHub is used to develop features, fix bugs, or experiment in isolation from the main project (often the "main" or "master" branch).
    \item \textbf{Commits:} Commits are changes saved to the repository. They record what changes were made and by whom.
    \item \textbf{Pull Requests (PRs):} Pull requests allow collaborators to review and discuss changes before they are merged into the main project branch.
    \item \textbf{Issues and Discussions:} GitHub provides a way to manage project tasks, bugs, and feature requests through issues and team collaboration via discussions.
    \item \textbf{Actions:} Automate your workflows with GitHub Actions, which allow continuous integration and deployment.
\end{itemize}

\vspace{20pt}
\begin{center}
\begin{tabular}{|p{0.40\linewidth}|}
    \hline
    \huge \textbf{What Is GitHub} \\
    \hline
    \end{tabular}
    \end{center}
    \vspace{20pt}
    
\textbf{GitHub Desktop} is a GUI (Graphical User Interface) client that allows you to manage your GitHub repositories easily without needing to use the command line. It simplifies Git commands like commits, pull requests, and merges by providing a user-friendly interface.

\vspace{10pt}

\subsection*{Benefits of GitHub Desktop:}
\begin{itemize}
    \item Easy visualization of changes, branches, and histories.
    \item Simplified process for committing, merging, and pushing code.
    \item Collaboration tools like pull requests are built-in.
    \item Useful for developers who are new to Git commands.
\end{itemize}

\vspace{20pt}

\subsection*{GitHub Desktop Installation Guide}
\textbf{Here are the steps to install GitHub Desktop:}
\begin{enumerate}
    \item \textbf{Download GitHub Desktop:}
    \begin{itemize}
        \item Visit \url{https://desktop.github.com} and click the Download for Windows or Download for macOS button, depending on your operating system.
    \end{itemize}
    \item \textbf{Run the Installer:}
    \begin{itemize}
        \item Once the download is complete, run the installation file to start the setup process.
        \item Follow the on-screen instructions to install GitHub Desktop on your computer.
    \end{itemize}
    \item \textbf{Sign in to GitHub:}
    \begin{itemize}
        \item After installation, launch GitHub Desktop and sign in using your GitHub account credentials.
        \item If you don’t have an account, you can sign up directly from the app or from \href{https://github.com}{GitHub's website}.
    \end{itemize}
\end{enumerate}

\newpage

% Placeholder for the image
\begin{center}
    \includegraphics[width=0.8\textwidth]{xyz2.png} \\ % Replace 'xyz2.png' with your actual image file name
    \vspace{20pt}
\end{center}

\noindent
\textbf{4. Clone or Create a Repository:}
\begin{itemize}
    \item Once signed in, you can clone a repository by entering the repository URL or selecting a repository from your GitHub account.
    \item You can also create a new repository directly from GitHub Desktop.
\end{itemize}

\vspace{260pt}
\begin{center}
\textbf{Lab Entry By}\\
Pronab Sarkar
\end{center}
\noindent\textbf{Assignment 2. Create a local repository. Build a C programme of calculator in the local repository, commit and publish it as a public repository}
\section*{Introduction}
This program is a simple calculator written in C, designed to perform basic arithmetic operations (addition, subtraction, multiplication, and division) based on user input. The user is prompted to enter two numbers, and then they choose an arithmetic operator (+, -, *, /) to perform the corresponding operation.

\section*{How the Program Works}
\begin{itemize}
    \item \textbf{Input:} The program first asks the user to input two floating-point numbers (decimal numbers) and stores them in the variables \texttt{x} and \texttt{y}.
    
    \item \textbf{Choosing an Operation:} The user is prompted to enter a symbol for the arithmetic operation they wish to perform:
    \begin{itemize}
        \item \texttt{+} for addition
        \item \texttt{-} for subtraction
        \item \texttt{*} for multiplication
        \item \texttt{/} for division
    \end{itemize}

    \item \textbf{Switch-Case Logic:} The program then uses a switch statement to determine which operation to perform based on the symbol entered by the user:
    \begin{itemize}
        \item If the user enters \texttt{+}, the program adds the two numbers and prints the result.
        \item If the user enters \texttt{-}, the program subtracts the second number from the first and prints the result.
        \item If the user enters \texttt{*}, the program multiplies the two numbers and prints the result.
        \item If the user enters \texttt{/}, the program divides the first number by the second and prints the result. (Note: Division by zero is not handled in this program).
    \end{itemize}
    
    \item \textbf{Error Handling:} If the user enters an invalid operator (anything other than \texttt{+}, \texttt{-}, \texttt{*}, or \texttt{/}), the program prints an error message and terminates.
\end{itemize}

\section*{Code Explanation}
\begin{verbatim}
#include <stdio.h>

int main() {
    float x, y;
    char ch;

    printf("Enter first number:\n");
    scanf("%f", &x);

    printf("Enter the second number:\n");
    scanf("%f", &y);

    printf("Enter\n+ for add\n- for sub\n* for multi\n/ for div\n");
    scanf(" %c", &ch);

    switch(ch) {
        case '+':
            printf("Add = %f\n", (x + y));
            break;
        case '-':
            printf("Sub = %f\n", (x - y));
            break;
        case '*':
            printf("Multi = %f\n", (x * y));
            break;
        case '/':
            if (y != 0)
                printf("Div = %f\n", (x / y));
            else
                printf("Error! Division by zero.\n");
            break;
        default:
            printf("Error! Operator is not correct.\n");
            return 1;
    }
    return 0;
}
\end{verbatim}

\section*{Function Explanations with Output Values}
\subsection*{1. Addition (+)}
If the user inputs \texttt{+}, the program adds \texttt{x} and \texttt{y} and prints the result.
\begin{verbatim}
Example Output:
Enter first number: 
5 
Enter the second number: 
3 
+ 
Add = 8.000000
\end{verbatim}

\subsection*{2. Subtraction (-)}
If the user inputs \texttt{-}, the program subtracts \texttt{y} from \texttt{x} and prints the result.
\begin{verbatim}
Example Output:
Enter first number: 
10 
Enter the second number: 
4 
- 
Sub = 6.000000
\end{verbatim}

\subsection*{3. Multiplication (*)}
If the user inputs \texttt{*}, the program multiplies \texttt{x} by \texttt{y} and prints the result.
\begin{verbatim}
Example Output:
Enter first number: 
7 
Enter the second number: 
2 
* 
Multi = 14.000000
\end{verbatim}

\subsection*{4. Division (/)}
If the user inputs \texttt{/}, the program divides \texttt{x} by \texttt{y} and prints the result, provided that \texttt{y} is not zero.
\begin{verbatim}
Example Output:
Enter first number: 
20 
Enter the second number: 
5 
/ 
Div = 4.000000
\end{verbatim}

If \texttt{y = 0}, the program will output an error message.
\begin{verbatim}
Example Output (Division by Zero):
Enter first number: 
10 
Enter the second number: 
0 
/ 
Error! Division by zero.
\end{verbatim}

\subsection*{5. Error Handling (Invalid Operator)}
If the user inputs an invalid operator (not \texttt{+}, \texttt{-}, \texttt{*}, or \texttt{/}), the program will display an error message.
\begin{verbatim}
Example Output:
Enter first number: 
8 
Enter the second number: 
2 
% 
Error! Operator is not correct.
\end{verbatim}

\section*{Conclusion}
This program demonstrates the basic use of arithmetic operations in C using a simple calculator interface. It covers user input, arithmetic operations, and basic error handling, providing a user-friendly experience.

\end{document}
